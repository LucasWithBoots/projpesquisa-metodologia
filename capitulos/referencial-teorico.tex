% ----------------------------------------------------------
% Fundamentação Teórica
% ----------------------------------------------------------
\chapter{Fundamentação Teórica}
O desenvolvimento de aplicativos móveis se tornou fundamental na era digital, com um aumento significativo no número de dispositivos móveis em todo o mundo \cite{statista:uso_celular}. Esse crescimento impulsionou o desenvolvimento de tecnologias que otimizam o processo de criação de aplicativos para as duas plataformas móveis mais populares: iOS e Android. No entanto, a criação de aplicativos para múltiplas plataformas apresenta desafios como compatibilidade, desempenho e manutenção de código, o que incentivou a criação de frameworks de desenvolvimento nativo e cross-platform para superar essas barreiras.

\section{Desenvolvimento Nativo}
Desenvolvimento nativo refere-se ao processo de criação de aplicativos usando linguagens e ferramentas específicas para cada sistema operacional. No caso do Android, as principais linguagens são Java e Kotlin, enquanto o iOS utiliza Swift e Objective-C. O desenvolvimento nativo é conhecido por oferecer melhor desempenho e integração com os recursos do dispositivo, pois o código é compilado diretamente para a plataforma de destino.

Por outro lado, o desenvolvimento cross-platform permite que os desenvolvedores escrevam um único código que funcione em múltiplas plataformas. Ferramentas como React Native, Flutter e Xamarin são populares nesse campo. Essa abordagem reduz o tempo e o custo de desenvolvimento, permitindo que equipes lancem produtos rapidamente. No entanto, frameworks cross-platform podem enfrentar limitações em relação ao desempenho e à personalização de interfaces, devido à necessidade de "pontes" para comunicação com componentes nativos.